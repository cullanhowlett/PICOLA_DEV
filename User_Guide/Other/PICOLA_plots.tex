\documentclass[12pt,twoside,a4paper]{article}
\usepackage[a4paper]{geometry}
\usepackage{graphicx}
\usepackage{epstopdf}
\usepackage{mathtools}
\usepackage{float}
\usepackage[small,bf]{caption}
\usepackage{subfig}
\usepackage{xfrac}
\usepackage{array}
\usepackage{fancyhdr}
\usepackage{longtable}
\setlength{\headheight}{15pt}
\pagestyle{fancy}

\providecommand{\incfig}[6]
{
\begin{figure}[h!]
   \centering
   \vspace{#1}
   \includegraphics[width=#2]{#3}
   \caption[#4]{#5}
   \label{#6}    
\end{figure}
}

\begin{document}

Below are some plots comparing halos from GADGET and PICOLA simulations run from the same initial conditions. The halos were created from dark matter particles using our Friends-of-friends code \texttt{CM\_HALOFINDER} with linking length 0.2 in units of the mean particle separation. The simulation was performed using $1024^3$ particles in a box of edge length $768 \text{h}^{-1}\text{Mpc}$, resulting in a particle mass of $\sim3.71\text{x}10^{10} \text{h}^{-1}\text{Mpc}$. Halos were matched under the assumption that if two halos were separated by a distance less than or equal to the sum of the virial radii as calculated from the PICOLA and GADGET masses then they are the same halo.
\begin{equation}
R_{vir} = \left( \frac{3M}{4\pi\Omega_{m}\rho_{c}\Delta_{vir}} \right)^{\frac{1}{3}}
\end{equation} 
We take the commonly used value of $\Delta{vir}=180$ as the overdensity at which a halo is defined. We bin the matched halos by GADGET mass and calculate the mean GADGET and PICOLA masses within that bin. The errors come from the variance in PICOLA masses within a given mass bin.

\incfig{0pt}{\textwidth}{CM_GADGET_PICOLA_MATCH_L768_N1024_R101_z0p00.png}{}{A plot of the GADGET halo mass vs. PICOLA halo mass for halos at z=0.0.}{FOFChart}

\incfig{0pt}{\textwidth}{CM_GADGET_PICOLA_MATCH_L768_N1024_R101_z0p50.png}{}{A plot of the GADGET halo mass vs. PICOLA halo mass for halos at z=0.5.}{FOFChart}

\incfig{0pt}{\textwidth}{CM_GADGET_PICOLA_MATCH_L768_N1024_R101_z1p00.png}{}{A plot of the GADGET halo mass vs. PICOLA halo mass for halos at z=1.0.}{FOFChart}

\end{document}